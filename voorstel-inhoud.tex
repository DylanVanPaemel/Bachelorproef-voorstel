%---------- Inleiding ---------------------------------------------------------

\section{Introductie} % The \section*{} command stops section numbering
\label{sec:introductie}
Technologie wordt iedere dag steeds belangrijker in ons leven. We gebruiken technologie om ons leven aangenamer en makkelijker te maken. Ook in gezonde levensstijl wordt technologie de dag vandaag veel gebruikt. Zowel in de Apple als de Play store zijn er veel applicaties die mobile coaching aanbieden, niet al van deze applicaties blijken allemaal kwaliteitsvol te zijn. \hfill \break \break
De bedoeling van dit onderzoek is om een voorstudie te doen voor applicatie ontwikkelaars in verband met mobile coaching. Dit zowel op gebruikersgemak als trainingsgebied zelf (zoals in onderzoek hier verwijzen + kleine toelichting over het onderzoek). Vele gebruikers zijn teleurgesteld in deze applicaties en bekomen niet het gewenste resultaat. Wat willen de gebruikers in deze applicaties om hen te helpen? Wat zijn absolute musts? Niet alleen functioneel maar hoe zien de gebruikers liefst hoe een mobile coaching applicatie is opgebouwd?
\end{itemize}

%---------- Stand van zaken ---------------------------------------------------

\section{literatuurstudie }
\label{sec:state-of-the-art}
In 2015 is een vergelijkende studie gemaakt namelijk ~\autocite{JMIR2015} omtrent mobile coaching applicaties die om dat moment op de markt waren. Het onderzoek werkt met een schaal waarin ze applicaties vergelijken met vooraf bepaalde richtlijnen. Uit het onderzoek blijkt dat slechts een klein procent aan alle richtlijnen voldoet om de gezondheid niet te schaden. In de applicaties staan de gebruikers vaak niet centraal en de applicaties zijn te commercieel ingericht. \break
\break
Mobile coaching is een veel ruimer begrip dan enkel maar bewegen en kracht oefeningen, voor een gezonde levensstijl is veel meer nodig. In ~\autocite{EQUILIBRIO2005} wordt ' life management system ' omschreven waarin kracht oefeningen en voedingsschema's centraal staan om een gezonde levensstijl te creëren. In het onderzoek gaat me er van uit dat je door dit te combineren een beter en gelukkiger leven zal hebben. \break
 \break
We kunnen dus vaststellen dat veel mobiele applicaties niet voldoen aan fundamentele aanbevelingen voor de gezondheid op vlak van beweging. Er zijn 3 goede indicators waar ontwikkelaars voor deze applicaties best rekening mee houden, namelijk: het FITT-principe\footnote{Frequency Intensity Type en Time’: hoe vaak , hoe intensief en hoe lang wordt er getraind?
}, veiligheid en structuur.

% Voor literatuurverwijzingen zijn er twee belangrijke commando's:
% \autocite{KEY} => (Auteur, jaartal) Gebruik dit als de naam van de auteu
%   geen onderdeel is van de zin.
% \textcite{KEY} => Auteur (jaartal)  Gebruik dit als de auteursnaam wel een
%   functie heeft in de zin (bv. ``Uit onderzoek door Doll & Hill (1954) bleek
%   ...'')



%---------- Methodologie ------------------------------------------------------
\section{Methodologie}
\label{sec:methodologie}
Dit onderzoek omtrent mobile coaching zal geen applicaties vergelijken maar gebruik maken van de community om te bepalen wat de gebruikers écht nodig hebben en wat niet. Daarom stellen we een vragenlijst op voor mobiele applicatie: 
\begin{itemize}
\item Wat is uw geslacht?
\item Wat is uw leeftijd?
\item Hoe belangrijk is het gebruiksgemak van een mobiele applicatie?
\item Hoe belangrijk is het uitzicht van een applicatie?
\item Gebruikt u een applicatie voor uw gezondheid?
\begin{itemize}
\item Indien niet zou u in de toekomst graag een applicatie gebruiken?
\end{itemize}
\item Wat wil u bijhouden in de applicatie:
\begin{itemize}
\item Wil u voedingswaarden bijhouden?
\item Wil u trainingschema's bijhouden?
\item Wil u illustraties bij oefeningen met de juiste voorgetoonde beweging? ( video's of afbeeldingen) 
\end{itemize}
\end{itemize}
Dit zijn een aantal indicatie vragen om aan te tonen welke richting dit onderzoek zal uitgaan, dit kan natuurlijk nog verder uitgebreid worden. De resultaten van de enquête zullen in een CSV-bestand worden gegoten die later kunnen worden verwerkt in RStudio. Daarom zal er gekozen worden om de enquête op te stellen in Google Forms, hieruit kan makkelijk een CSV-bestand worden gegenereerd.
 Als de gegevens eenmaal uitgewerkt zijn in Rstudio kunnen we beginnen met een rangschikking wat gebruikers belangrijk vinden en wat niet. \hfill \break \break
 De enquête zal zowel digitaal als op papier beschikbaar gesteld worden. De enquête zal gedeeld worden op Fora, Facebook groepen over gezondheid en in fitnesscentra.

%---------- Verwachte resultaten ----------------------------------------------
\section{Verwachte resultaten}
\label{sec:verwachte_resultaten}

Hier beschrijf je welke resultaten je verwacht. Als je metingen en simulaties uitvoert, kan je hier al mock-ups maken van de grafieken samen met de verwachte conclusies. Benoem zeker al je assen en de stukken van de grafiek die je gaat gebruiken. Dit zorgt ervoor dat je concreet weet hoe je je data gaat moeten structureren.

%---------- Verwachte conclusies ----------------------------------------------
\section{Verwachte conclusies}
\label{sec:verwachte_conclusies}

Hier beschrijf je wat je verwacht uit je onderzoek, met de motivatie waarom. Het is \textbf{niet} erg indien uit je onderzoek andere resultaten en conclusies vloeien dan dat je hier beschrijft: het is dan juist interessant om te onderzoeken waarom jouw hypothesen niet overeenkomen met de resultaten.

