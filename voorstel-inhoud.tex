%---------- Inleiding ---------------------------------------------------------

\section{Introductie} % The \section*{} command stops section numbering
\label{sec:introductie}
Technologie wordt iedere dag steeds belangrijker in ons leven. We gebruiken technologie onder andere om ons leven aangenamer en makkelijker te maken. Ook voor een gezonde levensstijl wordt technologie de dag vandaag veel gebruikt. Zowel in de Apple als de Play store zijn er veel applicaties die mobile coaching aanbieden, niet al van deze applicaties blijken allemaal kwaliteitsvol te zijn. \hfill \break \break
De bedoeling van dit onderzoek is om een voorstudie te doen voor applicatie ontwikkelaars in verband met mobile coaching. Dit zowel op gebruikersgemak als trainingsgebied. Vele gebruikers zijn teleurgesteld in deze applicaties en bekomen niet het gewenste resultaat. Wat willen de gebruikers in deze applicaties om hen te helpen met een gezonde levensstijl? Wat zijn absolute musts? Niet alleen functionaliteit is belangrijk maar hoeveel belang hechten mensen aan de user experience van deze applicaties?


%---------- Stand van zaken ---------------------------------------------------

\section{literatuurstudie }
\label{sec:state-of-the-art}
In 2015 is een vergelijkende studie gemaakt namelijk \textcite{JMIR2015}  omtrent mobile coaching applicaties die om dat moment op de markt waren. Het onderzoek werkt met een schaal waarin ze applicaties vergelijken met vooraf bepaalde richtlijnen. Uit het onderzoek blijkt dat slechts een klein procent aan alle richtlijnen voldoet om de gezondheid niet te schaden. In de applicaties staan de gebruikers vaak niet centraal en de applicaties zijn te commercieel ingericht. \break
\break
Mobile coaching is een veel ruimer begrip dan enkel maar bewegen en kracht oefeningen, voor een gezonde levensstijl is veel meer nodig. In \textcite{EQUILIBRIO2005}  wordt ' life management system ' omschreven waarin kracht oefeningen en voedingsschema's centraal staan om een gezonde levensstijl te creëren. In het onderzoek gaat me er van uit dat je door dit te combineren een beter en gelukkiger leven zal hebben. \break
 \break
We kunnen dus vaststellen dat veel mobiele applicaties niet voldoen aan fundamentele aanbevelingen voor de gezondheid op vlak van beweging. Er zijn 3 goede indicators waar ontwikkelaars voor deze applicaties best rekening mee houden, namelijk: het FITT-principe\footnote{Frequency Intensity Type en Time’: hoe vaak , hoe intensief en hoe lang wordt er getraind?
}, veiligheid en structuur.

% Voor literatuurverwijzingen zijn er twee belangrijke commando's:
% \autocite{KEY} => (Auteur, jaartal) Gebruik dit als de naam van de auteu
%   geen onderdeel is van de zin.
% \textcite{KEY} => Auteur (jaartal)  Gebruik dit als de auteursnaam wel een
%   functie heeft in de zin (bv. ``Uit onderzoek door Doll & Hill (1954) bleek
%   ...'')



%---------- Methodologie ------------------------------------------------------
\section{Methodologie}
\label{sec:methodologie}
Dit onderzoek omtrent mobile coaching zal geen applicaties vergelijken maar gebruik maken van de community om te bepalen wat de gebruikers écht nodig hebben en wat niet. Daarom stellen we een vragenlijst op voor een mobile coaching applicatie: 
\begin{itemize}
\item Wat is uw geslacht?
\item Wat is uw leeftijd?
\item Hoe belangrijk is het gebruiksgemak van een mobiele applicatie?
\item Hoe belangrijk is het uitzicht van een applicatie?
\item Gebruikt u een applicatie voor uw gezondheid?
\begin{itemize}
\item Indien niet zou u in de toekomst graag een applicatie gebruiken?
\end{itemize}
\item Wat wil u bijhouden in de applicatie:
\begin{itemize}
\item Wil u voedingswaarden bijhouden?
\item Wil u trainingsschema's bijhouden?
\item Wil u illustraties bij oefeningen met de juiste voorgetoonde beweging? ( video's of afbeeldingen) 
\end{itemize}
\end{itemize}
Dit zijn een aantal indicatie vragen om aan te tonen welke richting dit onderzoek zal uitgaan, dit kan natuurlijk nog verder uitgebreid worden en de vragenlijst is nog niet definitief. De resultaten van de enquête zullen in een CSV-bestand worden gegoten die later kunnen worden verwerkt in \textcite{RStudio} .
 Als de gegevens eenmaal uitgewerkt zijn in \textcite{RStudio} kunnen we onder andere  beginnen met een rangschikking wat gebruikers belangrijk vinden en wat niet. We kunnen ook verschillende types gebruikers gaan onderzoeken en een verband met hun antwoorden enzovoort.\hfill \break \break
 De enquête zal zowel digitaal als op papier beschikbaar gesteld worden. De enquête zal gedeeld worden op Fora, \textcite{Facebook} groepen over gezondheid en in bepaalde fitnesscentra.

%---------- Verwachte resultaten ----------------------------------------------
\section{Verwachte resultaten}
\label{sec:verwachte_resultaten}
Het zou zeker bevorderend zijn voor het onderzoek moeten we meer dan 100 resultaten mogen verwachten. Bij dit aantal kunnen we al een mooie conclusie vormen over wat de deelnemers van de enquête hebben geantwoord op de vragenlijst. \hfill \break \break
Er moet een duidelijk beeld kunnen gevormd worden van verschillende types gebruikers. Dit zal gebeuren enerzijds gebeuren aan de hand van de leeftijd en het geslacht en anderzijds zal er gekeken worden welke bedoeling de eind gebruiker heeft. Wil de eindgebruiker afvallen of spieren opbouwen? Heeft de gebruiker ervaring met voeding- en trainingsschema's? Dit kan een verschil zijn qua 'usergoal' en daarom wordt er verwacht dat er tussen deze groepen een duidelijk waarneembaar verschil zal zijn in de antwoorden op de enquête. Er moet ook gezorgd worden dat er voldoende deskundige personen de enquête afleggen, bijvoorbeeld dokters, personal coaches, diëtisten, enzovoort...\hfill \break \break
De verwachtingen van deze bachelorproef zijn dat dit in de toekomst kan gebruikt worden als basis voor een applicatie in verband met mobile coaching. 

%---------- Verwachte conclusies ----------------------------------------------
\section{Verwachte conclusies}
\label{sec:verwachte_conclusies}
De conclusie hangt natuurlijk af van de hoeveelheid antwoorden en de antwoorden zelf op onze vragenlijst. Verder wordt er verwacht dat de gebruikers 'user experience' boven de functionaliteit verkiezen. Iets wat professioneel oogt zal meestal als goed in het algemeen bestempeld worden. \hfill \break \break Er wordt ook verwacht dat de gebruikers graag met verschillende niveaus willen werken. Bijvoorbeeld kan er een keuze gemaakt worden tussen beginner of gevorderde gebruiker en afhankelijk van die keuze en andere gegevens van de gebruiker, zoals leeftijd en gewicht, zal de applicatie gepaste oefeningen en trainingsschema's genereren. Op deze manier zal iedere gebruiker zich comfortabel voelen bij elke training en maaltijd en dat is tenslotte allemaal waar het om draait.

