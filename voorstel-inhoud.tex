%---------- Inleiding ---------------------------------------------------------

\section{Introductie} % The \section*{} command stops section numbering
\label{sec:introductie}

De bedoeling van dit onderzoek is om een voorstudie te doen voor applicatie ontwikkelaars in verband met mobile coaching. Zowel in de Apple als de Play store zijn er veel applicaties die mobile coaching aanbieden, niet al van deze applicaties blijken allemaal kwaliteitsvol te zijn. Dit zowel op gebruikersgemak als trainingsgebied zelf (zoals in onderzoek hier verwijzen + kleine toelichting over het onderzoek). Vele gebruikers zijn teleurgesteld in deze applicaties en bekomen niet het gewenste resultaat. Wat willen de gebruikers in deze applicaties om hen te helpen? Wat zijn absolute musts? Niet alleen functioneel maar hoe zien de gebruikers liefst hoe een mobile coaching applicatie is opgebouwd?
\end{itemize}

%---------- Stand van zaken ---------------------------------------------------

\section{literatuurstudie }
\label{sec:state-of-the-art}
In 2015 is een vergelijkende studie gemaakt namelijk in ~\autocite{JMIR2015} omtrent mobile coaching applicaties die om dat moment op de markt waren. Het onderzoek werkt met een schaal waarin ze applicaties vergelijken met vooraf bepaalde richtlijnen. Uit het onderzoek blijkt dat slechts een klein procent aan alle richtlijnen voldoet om de gezondheid niet te schaden. In de applicaties staan de gebruikers vaak niet centraal en de applicaties zijn te commercieel ingericht. \break
\break
Mobile coaching is een veel ruimer begrip dan enkel maar bewegen en kracht oefeningen, voor een gezonde levensstijl is veel meer nodig. In ~\autocite{EQUILIBRIO2005} wordt ' life management system ' omschreven waarin kracht oefeningen en voedingsschema's centraal staan om een gezonde levensstijl te creëren. In het onderzoek gaat me er van uit dat je door dit te combineren een beter en gelukkiger leven zal hebben. \break


% Voor literatuurverwijzingen zijn er twee belangrijke commando's:
% \autocite{KEY} => (Auteur, jaartal) Gebruik dit als de naam van de auteu
%   geen onderdeel is van de zin.
% \textcite{KEY} => Auteur (jaartal)  Gebruik dit als de auteursnaam wel een
%   functie heeft in de zin (bv. ``Uit onderzoek door Doll & Hill (1954) bleek
%   ...'')



%---------- Methodologie ------------------------------------------------------
\section{Methodologie}
\label{sec:methodologie}

Dit onderzoek omtrent mobile coaching zal geen applicaties vergelijken maar gebruik maken van de community om te bepalen wat de gebruikers écht nodig hebben en wat niet. \break
\break
Hier beschrijf je hoe je van plan bent het onderzoek te voeren. Welke onderzoekstechniek ga je toepassen om elk van je onderzoeksvragen te beantwoorden? Gebruik je hiervoor experimenten, vragenlijsten, simulaties? Je beschrijft ook al welke tools je denkt hiervoor te gebruiken of te ontwikkelen.

%---------- Verwachte resultaten ----------------------------------------------
\section{Verwachte resultaten}
\label{sec:verwachte_resultaten}

Hier beschrijf je welke resultaten je verwacht. Als je metingen en simulaties uitvoert, kan je hier al mock-ups maken van de grafieken samen met de verwachte conclusies. Benoem zeker al je assen en de stukken van de grafiek die je gaat gebruiken. Dit zorgt ervoor dat je concreet weet hoe je je data gaat moeten structureren.

%---------- Verwachte conclusies ----------------------------------------------
\section{Verwachte conclusies}
\label{sec:verwachte_conclusies}

Hier beschrijf je wat je verwacht uit je onderzoek, met de motivatie waarom. Het is \textbf{niet} erg indien uit je onderzoek andere resultaten en conclusies vloeien dan dat je hier beschrijft: het is dan juist interessant om te onderzoeken waarom jouw hypothesen niet overeenkomen met de resultaten.

