\documentclass[fleqn,10pt]{voorstel}

%------------------------------------------------------------------------------
% Metadata over het voorstel
%------------------------------------------------------------------------------

\JournalInfo{HoGent Bedrijf en Organisatie}
\Archive{Bachelorproef 2018 - 2019} % Of: Onderzoekstechnieken

%---------- Titel & auteur ----------------------------------------------------

% TODO: geef werktitel van je eigen voorstel op
\PaperTitle{Mobile coaching applicatie: functionaliteiten en user experience}
\PaperType{Onderzoeksvoorstel Bachelorproef} % Type document

% TODO: vul je eigen naam in als auteur, geef ook je emailadres mee!
\Authors{Dylan Van Paemel\textsuperscript{1}} % Authors
\CoPromotor{Nog in orde te brengen\textsuperscript{2}}
\affiliation{\textbf{Contact:}
  \textsuperscript{1} \href{mailto:dylan.vanpaemel.w1044@student.hogent.be}{dylan.vanpaemel.w1044@student.hogent.be};
}

%---------- Abstract ----------------------------------------------------------

\Abstract{Technologie is niet meer weg te denken uit ons leven, ook in de gezondheidssector wordt er tegenwoordig veel gebruik gemaakt van technologie. Er zijn honderden applicaties beschikbaar in de Apple en Play store die mobile coaching aanbieden, echter blijkt uit onderzoek dat slechts een klein aantal van deze applicaties kwaliteitsvol zijn. In dit onderzoek zal er gebruikt gemaakt worden van de community om een onderzoek te doen naar de belangrijkste noden in een mobile coaching applicatie door middel van een vragenlijst. In dit onderzoek zal onderzocht worden als er een type applicatie kan gekoppeld worden aan een bepaalde gebruiker, met andere woorden kunnen we aan de hand van vragen de keuze van een applicatie voorspellen of niet? De resultaten van dit onderzoek kunnen tevens een goeie basis vormen voor de ontwikkelaars omdat we door de vragenlijst een duidelijk beeld zullen krijgen wat mensen in een coaching applicatie verwachten en waarom gebaseerd moet worden.
}

%---------- Onderzoeksdomein en sleutelwoorden --------------------------------
% TODO: Sleutelwoorden:
%
% Het eerste sleutelwoord beschrijft het onderzoeksdomein. Je kan kiezen uit
% deze lijst:
%
% - Mobiele applicatieontwikkeling
% - Webapplicatieontwikkeling
% - Applicatieontwikkeling (andere)
% - Systeembeheer
% - Netwerkbeheer
% - Mainframe
% - E-business
% - Databanken en big data
% - Machineleertechnieken en kunstmatige intelligentie
% - Andere (specifieer)
%
% De andere sleutelwoorden zijn vrij te kiezen

\Keywords{Mobiele applicatie --- Fitness --- Machine learning} % Keywords
\newcommand{\keywordname}{Sleutelwoorden} % Defines the keywords heading name

%---------- Titel, inhoud -----------------------------------------------------

\begin{document}

\flushbottom % Makes all text pages the same height
\maketitle % Print the title and abstract box
\tableofcontents % Print the contents section
\thispagestyle{empty} % Removes page numbering from the first page

%------------------------------------------------------------------------------
% Hoofdtekst
%------------------------------------------------------------------------------

% De hoofdtekst van het voorstel zit in een apart bestand, zodat het makkelijk
% kan opgenomen worden in de bijlagen van de bachelorproef zelf.
%---------- Inleiding ---------------------------------------------------------

\section{Introductie} % The \section*{} command stops section numbering
\label{sec:introductie}
Technologie wordt iedere dag steeds belangrijker in ons leven. We gebruiken technologie onder andere om ons leven aangenamer te maken. Zowel in de Apple als de Play store zijn er veel applicaties die mobile coaching aanbieden, het doel van deze applicaties is om een gezondere levensstijl te hanteren. Uit onderzoek blijkt dat niet alle applicaties even kwaliteitsvol zijn. \hfill \break \break
De bedoeling van dit onderzoek is om een voorstudie te doen voor applicatie ontwikkelaars om betere mobile coaching applicaties te bouwen met behulp van de community.Want veel gebruikers zijn teleurgesteld in deze applicaties en bekomen niet hun gewenst resultaat. Wat willen de gebruikers in deze applicaties? Wat zijn absolute musts? En kunnen we aan de hand van de antwoorden voorspellen welke applicatie bij welke gebruiker past met behulp van machine learning?


%---------- Stand van zaken ---------------------------------------------------

\section{literatuurstudie }
\label{sec:state-of-the-art}
In 2015 is een vergelijkende studie gedaan, namelijk \textcite{JMIR2015}. Het onderzoek bestudeerde mobile coaching applicaties die om dat moment op de markt waren. Er wordt gewerkt met een schaal waarin men applicaties vergelijken met vooraf vastgelegde richtlijnen. Uit de conclusie blijkt dat slechts een klein procent aan alle richtlijnen voldoet. Het is belangrijk bij dit type applicaties dat de gebruiker centraal staat. \break\break
Onder de noemer mobile coaching valt meer dan bewegingsoefeningen, voor een gezonde levensstijl is veel meer nodig. In \textcite{EQUILIBRIO2005}  wordt ' life management system ' omschreven waarin krachtoefeningen en voedingsschema's centraal staan om een gezonde levensstijl te evenaren.\textcite{EQUILIBRIO2005} beschrijft dat je door deze combinatie een gelukkiger leven zal leiden. \break
 \break
We stellen dus vast dat veel mobiele applicaties niet voldoen aan fundamentele aanbevelingen voor de gezondheid. Er zijn 3 goede indicators waar ontwikkelaars voor dit type applicaties best rekening mee houden, namelijk: het FITT-principe\footnote{Frequency Intensity Type en Time’: hoe vaak , hoe intensief en hoe lang wordt er getraind?
}, veiligheid en structuur.

% Voor literatuurverwijzingen zijn er twee belangrijke commando's:
% \autocite{KEY} => (Auteur, jaartal) Gebruik dit als de naam van de auteu
%   geen onderdeel is van de zin.
% \textcite{KEY} => Auteur (jaartal)  Gebruik dit als de auteursnaam wel een
%   functie heeft in de zin (bv. ``Uit onderzoek door Doll & Hill (1954) bleek
%   ...'')



%---------- Methodologie ------------------------------------------------------
\section{Methodologie}
\label{sec:methodologie}
Dit onderzoek omtrent mobile coaching zal geen applicaties vergelijken maar gebruik maken van de community om te bepalen wat de gebruikers écht nodig hebben en wat niet. Daarom stellen we een vragenlijst op voor een mobile coaching applicatie: 
\begin{itemize}
\item Wat is uw geslacht?
\item Wat is uw leeftijd?
\item Hoe vaak sport u per week?
\item Hoe beoefent u meestal lichaamsbeweging?
\item Welk besturingssysteem heeft uw smartphone?
\item Hoe belangrijk vindt u het uitzicht en gebruiksgemak van een applicatie?
\item Heeft u al betalende applicaties gedownload?
\item Heeft u al een mobile coaching applicatie gebruikt?
\item Vindt u reclame in applicaties storend?
\item Ik kies voor een Mobile Coaching applicatie waarin ik...  
\begin{itemize}
\item enkel voedingsschema's krijg.
\item enkel trainingsschema's krijg.
\item voeding- en trainingschema's krijg.
\end{itemize}
\item Wat is uw hoofddoel bij het gebruik van een Mobile Coaching applicatie?
\item Ik vind extra uitleg of filmpjes bij oefeningen...?
\item Welke applicatie past best bij mij?
\end{itemize}
 De vragenlijst zal hoofdzakelijk digitaal beschikbaar gesteld worden op \textcite{Facebook} en gezondheidsfora.  Voor het onderzoek zal zelfgemaakte een PHP-omgeving voorzien worden waarin de antwoorden opgeslagen worden in een CSV-bestand. \hfill \break \break 
 In het tweede deel van het onderzoek zal een \textcite{Python} omgeving opgezet worden om machine learning toe te passen. Er zal een model met neuronen getraind worden met \textcite{Keras} en onderliggend \textcite{TensorFlow}. Het model zal dagelijks 2 keer getraind worden met data. Eens het model voldoende getraind is, zal het in staat zijn om voorspellingen te doen met nieuwe data.

%---------- Verwachte resultaten ----------------------------------------------
\section{Verwachte resultaten}
\label{sec:verwachte_resultaten}
Om correcte voorspellingen te doen met machine learning zijn veel resultaten noodzakelijk. Hoe meer antwoorden we krijgen uit onze vragenlijst, hoe accurater onze machine een voorspelling zal kunnen doen. Indien er niet genoeg resultaten zijn zal alles toch uitgewerkt worden maar dan als proof-of-concept.
\hfill \break \break


%---------- Verwachte conclusies ----------------------------------------------
\section{Verwachte conclusies}
\label{sec:verwachte_conclusies}
De conclusie hangt af van het aantal deelnemers aan het onderzoek. Verder wordt er verwacht dat de gebruikers 'user experience' boven de functionaliteit verkiezen. Iets wat professioneel oogt zal meestal als goed in het algemeen bestempeld worden. Er wordt verwacht dat men duidelijk gebruikers kan onderverdelen in verschillende groepen aan de hand de vragenlijst. \hfill \break \break 



%------------------------------------------------------------------------------
% Referentielijst
%------------------------------------------------------------------------------
% TODO: de gerefereerde werken moeten in BibTeX-bestand ``voorstel.bib''
% voorkomen. Gebruik JabRef om je bibliografie bij te houden en vergeet niet
% om compatibiliteit met Biber/BibLaTeX aan te zetten (File > Switch to
% BibLaTeX mode)

\phantomsection
\printbibliography[heading=bibintoc]

\end{document}
